% Options for packages loaded elsewhere
\PassOptionsToPackage{unicode}{hyperref}
\PassOptionsToPackage{hyphens}{url}
%
\documentclass[
]{article}
\usepackage{amsmath,amssymb}
\usepackage{iftex}
\ifPDFTeX
  \usepackage[T1]{fontenc}
  \usepackage[utf8]{inputenc}
  \usepackage{textcomp} % provide euro and other symbols
\else % if luatex or xetex
  \usepackage{unicode-math} % this also loads fontspec
  \defaultfontfeatures{Scale=MatchLowercase}
  \defaultfontfeatures[\rmfamily]{Ligatures=TeX,Scale=1}
\fi
\usepackage{lmodern}
\ifPDFTeX\else
  % xetex/luatex font selection
\fi
% Use upquote if available, for straight quotes in verbatim environments
\IfFileExists{upquote.sty}{\usepackage{upquote}}{}
\IfFileExists{microtype.sty}{% use microtype if available
  \usepackage[]{microtype}
  \UseMicrotypeSet[protrusion]{basicmath} % disable protrusion for tt fonts
}{}
\makeatletter
\@ifundefined{KOMAClassName}{% if non-KOMA class
  \IfFileExists{parskip.sty}{%
    \usepackage{parskip}
  }{% else
    \setlength{\parindent}{0pt}
    \setlength{\parskip}{6pt plus 2pt minus 1pt}}
}{% if KOMA class
  \KOMAoptions{parskip=half}}
\makeatother
\usepackage{xcolor}
\usepackage[margin=1in]{geometry}
\usepackage{color}
\usepackage{fancyvrb}
\newcommand{\VerbBar}{|}
\newcommand{\VERB}{\Verb[commandchars=\\\{\}]}
\DefineVerbatimEnvironment{Highlighting}{Verbatim}{commandchars=\\\{\}}
% Add ',fontsize=\small' for more characters per line
\usepackage{framed}
\definecolor{shadecolor}{RGB}{248,248,248}
\newenvironment{Shaded}{\begin{snugshade}}{\end{snugshade}}
\newcommand{\AlertTok}[1]{\textcolor[rgb]{0.94,0.16,0.16}{#1}}
\newcommand{\AnnotationTok}[1]{\textcolor[rgb]{0.56,0.35,0.01}{\textbf{\textit{#1}}}}
\newcommand{\AttributeTok}[1]{\textcolor[rgb]{0.13,0.29,0.53}{#1}}
\newcommand{\BaseNTok}[1]{\textcolor[rgb]{0.00,0.00,0.81}{#1}}
\newcommand{\BuiltInTok}[1]{#1}
\newcommand{\CharTok}[1]{\textcolor[rgb]{0.31,0.60,0.02}{#1}}
\newcommand{\CommentTok}[1]{\textcolor[rgb]{0.56,0.35,0.01}{\textit{#1}}}
\newcommand{\CommentVarTok}[1]{\textcolor[rgb]{0.56,0.35,0.01}{\textbf{\textit{#1}}}}
\newcommand{\ConstantTok}[1]{\textcolor[rgb]{0.56,0.35,0.01}{#1}}
\newcommand{\ControlFlowTok}[1]{\textcolor[rgb]{0.13,0.29,0.53}{\textbf{#1}}}
\newcommand{\DataTypeTok}[1]{\textcolor[rgb]{0.13,0.29,0.53}{#1}}
\newcommand{\DecValTok}[1]{\textcolor[rgb]{0.00,0.00,0.81}{#1}}
\newcommand{\DocumentationTok}[1]{\textcolor[rgb]{0.56,0.35,0.01}{\textbf{\textit{#1}}}}
\newcommand{\ErrorTok}[1]{\textcolor[rgb]{0.64,0.00,0.00}{\textbf{#1}}}
\newcommand{\ExtensionTok}[1]{#1}
\newcommand{\FloatTok}[1]{\textcolor[rgb]{0.00,0.00,0.81}{#1}}
\newcommand{\FunctionTok}[1]{\textcolor[rgb]{0.13,0.29,0.53}{\textbf{#1}}}
\newcommand{\ImportTok}[1]{#1}
\newcommand{\InformationTok}[1]{\textcolor[rgb]{0.56,0.35,0.01}{\textbf{\textit{#1}}}}
\newcommand{\KeywordTok}[1]{\textcolor[rgb]{0.13,0.29,0.53}{\textbf{#1}}}
\newcommand{\NormalTok}[1]{#1}
\newcommand{\OperatorTok}[1]{\textcolor[rgb]{0.81,0.36,0.00}{\textbf{#1}}}
\newcommand{\OtherTok}[1]{\textcolor[rgb]{0.56,0.35,0.01}{#1}}
\newcommand{\PreprocessorTok}[1]{\textcolor[rgb]{0.56,0.35,0.01}{\textit{#1}}}
\newcommand{\RegionMarkerTok}[1]{#1}
\newcommand{\SpecialCharTok}[1]{\textcolor[rgb]{0.81,0.36,0.00}{\textbf{#1}}}
\newcommand{\SpecialStringTok}[1]{\textcolor[rgb]{0.31,0.60,0.02}{#1}}
\newcommand{\StringTok}[1]{\textcolor[rgb]{0.31,0.60,0.02}{#1}}
\newcommand{\VariableTok}[1]{\textcolor[rgb]{0.00,0.00,0.00}{#1}}
\newcommand{\VerbatimStringTok}[1]{\textcolor[rgb]{0.31,0.60,0.02}{#1}}
\newcommand{\WarningTok}[1]{\textcolor[rgb]{0.56,0.35,0.01}{\textbf{\textit{#1}}}}
\usepackage{graphicx}
\makeatletter
\def\maxwidth{\ifdim\Gin@nat@width>\linewidth\linewidth\else\Gin@nat@width\fi}
\def\maxheight{\ifdim\Gin@nat@height>\textheight\textheight\else\Gin@nat@height\fi}
\makeatother
% Scale images if necessary, so that they will not overflow the page
% margins by default, and it is still possible to overwrite the defaults
% using explicit options in \includegraphics[width, height, ...]{}
\setkeys{Gin}{width=\maxwidth,height=\maxheight,keepaspectratio}
% Set default figure placement to htbp
\makeatletter
\def\fps@figure{htbp}
\makeatother
\setlength{\emergencystretch}{3em} % prevent overfull lines
\providecommand{\tightlist}{%
  \setlength{\itemsep}{0pt}\setlength{\parskip}{0pt}}
\setcounter{secnumdepth}{-\maxdimen} % remove section numbering
\ifLuaTeX
  \usepackage{selnolig}  % disable illegal ligatures
\fi
\IfFileExists{bookmark.sty}{\usepackage{bookmark}}{\usepackage{hyperref}}
\IfFileExists{xurl.sty}{\usepackage{xurl}}{} % add URL line breaks if available
\urlstyle{same}
\hypersetup{
  pdftitle={Calculate\_AUC\_unbalanceZ},
  pdfauthor={Xuhong},
  hidelinks,
  pdfcreator={LaTeX via pandoc}}

\title{Calculate\_AUC\_unbalanceZ}
\author{Xuhong}
\date{2023-12-09}

\begin{document}
\maketitle

\begin{Shaded}
\begin{Highlighting}[]
\CommentTok{\#Add X2, X1, Z variables to explore the impact of Z imbalance on AUC}
\FunctionTok{library}\NormalTok{(msm)}
\end{Highlighting}
\end{Shaded}

\begin{verbatim}
## Warning: 程辑包'msm'是用R版本4.3.2 来建造的
\end{verbatim}

\begin{Shaded}
\begin{Highlighting}[]
\FunctionTok{library}\NormalTok{(pROC)}
\end{Highlighting}
\end{Shaded}

\begin{verbatim}
## Warning: 程辑包'pROC'是用R版本4.3.2 来建造的
\end{verbatim}

\begin{verbatim}
## Type 'citation("pROC")' for a citation.
\end{verbatim}

\begin{verbatim}
## 
## 载入程辑包:'pROC'
\end{verbatim}

\begin{verbatim}
## The following objects are masked from 'package:stats':
## 
##     cov, smooth, var
\end{verbatim}

\begin{Shaded}
\begin{Highlighting}[]
\NormalTok{beta0 }\OtherTok{\textless{}{-}} \DecValTok{4}
\NormalTok{beta\_X }\OtherTok{\textless{}{-}} \FunctionTok{c}\NormalTok{(}\FloatTok{0.6}\NormalTok{, }\FloatTok{0.4}\NormalTok{, }\FloatTok{0.3}\NormalTok{)}
\FunctionTok{set.seed}\NormalTok{(}\DecValTok{2023}\NormalTok{)}
\NormalTok{NN }\OtherTok{\textless{}{-}} \DecValTok{3000}
\NormalTok{prevalence\_Z }\OtherTok{\textless{}{-}} \FunctionTok{seq}\NormalTok{(}\FloatTok{0.1}\NormalTok{, }\FloatTok{0.5}\NormalTok{, }\FloatTok{0.1}\NormalTok{)}

\NormalTok{ORZ\_values}\OtherTok{\textless{}{-}} \DecValTok{1}\SpecialCharTok{:}\DecValTok{5}  \CommentTok{\# Set different ORZ values }

\NormalTok{Result.AUC }\OtherTok{\textless{}{-}} \FunctionTok{matrix}\NormalTok{(}\DecValTok{0}\NormalTok{, }\AttributeTok{nrow =} \FunctionTok{length}\NormalTok{(prevalence\_Z) }\SpecialCharTok{*} \FunctionTok{length}\NormalTok{(ORZ\_values), }\AttributeTok{ncol =} \DecValTok{6}\NormalTok{)}
\FunctionTok{colnames}\NormalTok{(Result.AUC) }\OtherTok{\textless{}{-}} \FunctionTok{c}\NormalTok{(}\StringTok{"prevalence\_Z"}\NormalTok{, }\StringTok{"ORZ"}\NormalTok{, }\StringTok{"AUC\_model1"}\NormalTok{, }\StringTok{"AUC\_model2"}\NormalTok{, }\StringTok{"count\_z\_1"}\NormalTok{,}\StringTok{"count\_z\_0"}\NormalTok{)}

\NormalTok{row\_counter }\OtherTok{\textless{}{-}} \DecValTok{1}


\ControlFlowTok{for}\NormalTok{ (i }\ControlFlowTok{in} \DecValTok{1}\SpecialCharTok{:}\FunctionTok{length}\NormalTok{(prevalence\_Z)) \{}
  \ControlFlowTok{for}\NormalTok{ (j }\ControlFlowTok{in} \DecValTok{1}\SpecialCharTok{:}\FunctionTok{length}\NormalTok{(ORZ\_values)) \{}
\NormalTok{    ORZ }\OtherTok{\textless{}{-}}\NormalTok{ ORZ\_values[j]}
\NormalTok{    beta\_Z }\OtherTok{\textless{}{-}} \FunctionTok{log}\NormalTok{(ORZ)}
    
\NormalTok{    Result.AUC[row\_counter, }\DecValTok{1}\NormalTok{] }\OtherTok{\textless{}{-}}\NormalTok{ prevalence\_Z[i]}
\NormalTok{    Result.AUC[row\_counter, }\DecValTok{2}\NormalTok{] }\OtherTok{\textless{}{-}}\NormalTok{ ORZ}
    
\NormalTok{    X1 }\OtherTok{\textless{}{-}} \FunctionTok{rnorm}\NormalTok{(NN, }\DecValTok{0}\NormalTok{, }\DecValTok{1}\NormalTok{)}
\NormalTok{    X2 }\OtherTok{\textless{}{-}} \FunctionTok{rnorm}\NormalTok{(NN, }\DecValTok{0}\NormalTok{, }\DecValTok{1}\NormalTok{)}
\NormalTok{    X3 }\OtherTok{\textless{}{-}} \FunctionTok{rpois}\NormalTok{(NN, }\FloatTok{0.3}\NormalTok{)}
    
    \CommentTok{\# Generate X variable. At this point, Z is independent of X and is generated independently}
\NormalTok{    Z }\OtherTok{\textless{}{-}} \FunctionTok{sample}\NormalTok{(}\FunctionTok{c}\NormalTok{(}\DecValTok{0}\NormalTok{, }\DecValTok{1}\NormalTok{), NN, }\AttributeTok{replace =} \ConstantTok{TRUE}\NormalTok{, }\AttributeTok{prob =} \FunctionTok{c}\NormalTok{(}\DecValTok{1} \SpecialCharTok{{-}}\NormalTok{ prevalence\_Z[i], prevalence\_Z[i]))}
    
\NormalTok{    count\_z\_1 }\OtherTok{\textless{}{-}} \FunctionTok{sum}\NormalTok{(Z }\SpecialCharTok{==} \DecValTok{1}\NormalTok{)}
\NormalTok{    count\_z\_0 }\OtherTok{\textless{}{-}} \FunctionTok{sum}\NormalTok{(Z }\SpecialCharTok{==} \DecValTok{0}\NormalTok{)}
    
\NormalTok{    Result.AUC[row\_counter, }\DecValTok{5}\NormalTok{] }\OtherTok{\textless{}{-}}\NormalTok{ count\_z\_1}
\NormalTok{    Result.AUC[row\_counter, }\DecValTok{6}\NormalTok{] }\OtherTok{\textless{}{-}}\NormalTok{ count\_z\_0}
    
    
\NormalTok{    tmp }\OtherTok{\textless{}{-}} \FunctionTok{cbind}\NormalTok{(}\DecValTok{1}\NormalTok{, X1, X2, X3, Z) }\SpecialCharTok{\%*\%} \FunctionTok{c}\NormalTok{(beta0, beta\_X, beta\_Z)}
\NormalTok{    tmp }\OtherTok{\textless{}{-}} \FunctionTok{exp}\NormalTok{(tmp)}
\NormalTok{    y\_true }\OtherTok{\textless{}{-}} \FunctionTok{rbinom}\NormalTok{(NN, }\AttributeTok{size =} \DecValTok{1}\NormalTok{, }\AttributeTok{prob =}\NormalTok{ tmp }\SpecialCharTok{/}\NormalTok{ (}\DecValTok{1} \SpecialCharTok{+}\NormalTok{ tmp))}
    
\NormalTok{    Dat.validation }\OtherTok{\textless{}{-}} \FunctionTok{data.frame}\NormalTok{(}\AttributeTok{Y =}\NormalTok{ y\_true, }\AttributeTok{X1 =}\NormalTok{ X1, }\AttributeTok{X2 =}\NormalTok{ X2, }\AttributeTok{X3 =}\NormalTok{ X3, }\AttributeTok{Z =}\NormalTok{ Z)}
\NormalTok{    model1 }\OtherTok{\textless{}{-}} \FunctionTok{glm}\NormalTok{(Y }\SpecialCharTok{\textasciitilde{}}\NormalTok{ X1 }\SpecialCharTok{+}\NormalTok{ X2 }\SpecialCharTok{+}\NormalTok{ X3, }\AttributeTok{family =} \StringTok{"binomial"}\NormalTok{, }\AttributeTok{data =}\NormalTok{ Dat.validation)}
\NormalTok{    model2 }\OtherTok{\textless{}{-}} \FunctionTok{glm}\NormalTok{(Y }\SpecialCharTok{\textasciitilde{}}\NormalTok{ X1 }\SpecialCharTok{+}\NormalTok{ X2 }\SpecialCharTok{+}\NormalTok{ X3 }\SpecialCharTok{+}\NormalTok{ Z, }\AttributeTok{family =} \StringTok{"binomial"}\NormalTok{, }\AttributeTok{data =}\NormalTok{ Dat.validation)}
    
    \FunctionTok{summary}\NormalTok{(model1)}
    \FunctionTok{summary}\NormalTok{(model2)}
    
    
\NormalTok{    Est\_Est1 }\OtherTok{\textless{}{-}} \FunctionTok{as.numeric}\NormalTok{(model1}\SpecialCharTok{$}\NormalTok{coefficients)}
\NormalTok{    Predicted.Phi1 }\OtherTok{\textless{}{-}}\NormalTok{ model1}\SpecialCharTok{$}\NormalTok{fitted.values}
\NormalTok{    roc\_obj1 }\OtherTok{\textless{}{-}} \FunctionTok{roc}\NormalTok{(y\_true, Predicted.Phi1)}
\NormalTok{    Result.AUC[row\_counter, }\DecValTok{3}\NormalTok{] }\OtherTok{\textless{}{-}} \FunctionTok{auc}\NormalTok{(roc\_obj1)}
    
\NormalTok{    Est\_Est2 }\OtherTok{\textless{}{-}} \FunctionTok{as.numeric}\NormalTok{(model2}\SpecialCharTok{$}\NormalTok{coefficients)}
\NormalTok{    Predicted.Phi2 }\OtherTok{\textless{}{-}}\NormalTok{ model2}\SpecialCharTok{$}\NormalTok{fitted.values}
\NormalTok{    roc\_obj2 }\OtherTok{\textless{}{-}} \FunctionTok{roc}\NormalTok{(y\_true, Predicted.Phi2)}
\NormalTok{    Result.AUC[row\_counter, }\DecValTok{4}\NormalTok{] }\OtherTok{\textless{}{-}} \FunctionTok{auc}\NormalTok{(roc\_obj2)}
    
\NormalTok{    row\_counter }\OtherTok{\textless{}{-}}\NormalTok{ row\_counter }\SpecialCharTok{+} \DecValTok{1}
\NormalTok{  \}}
\NormalTok{\}}
\end{Highlighting}
\end{Shaded}

\begin{verbatim}
## Setting levels: control = 0, case = 1
\end{verbatim}

\begin{verbatim}
## Setting direction: controls < cases
\end{verbatim}

\begin{verbatim}
## Setting levels: control = 0, case = 1
\end{verbatim}

\begin{verbatim}
## Setting direction: controls < cases
\end{verbatim}

\begin{verbatim}
## Setting levels: control = 0, case = 1
\end{verbatim}

\begin{verbatim}
## Setting direction: controls < cases
\end{verbatim}

\begin{verbatim}
## Setting levels: control = 0, case = 1
\end{verbatim}

\begin{verbatim}
## Setting direction: controls < cases
\end{verbatim}

\begin{verbatim}
## Setting levels: control = 0, case = 1
\end{verbatim}

\begin{verbatim}
## Setting direction: controls < cases
\end{verbatim}

\begin{verbatim}
## Setting levels: control = 0, case = 1
\end{verbatim}

\begin{verbatim}
## Setting direction: controls < cases
\end{verbatim}

\begin{verbatim}
## Setting levels: control = 0, case = 1
\end{verbatim}

\begin{verbatim}
## Setting direction: controls < cases
\end{verbatim}

\begin{verbatim}
## Setting levels: control = 0, case = 1
\end{verbatim}

\begin{verbatim}
## Setting direction: controls < cases
\end{verbatim}

\begin{verbatim}
## Setting levels: control = 0, case = 1
\end{verbatim}

\begin{verbatim}
## Setting direction: controls < cases
\end{verbatim}

\begin{verbatim}
## Setting levels: control = 0, case = 1
\end{verbatim}

\begin{verbatim}
## Setting direction: controls < cases
\end{verbatim}

\begin{verbatim}
## Setting levels: control = 0, case = 1
\end{verbatim}

\begin{verbatim}
## Setting direction: controls < cases
\end{verbatim}

\begin{verbatim}
## Setting levels: control = 0, case = 1
\end{verbatim}

\begin{verbatim}
## Setting direction: controls < cases
\end{verbatim}

\begin{verbatim}
## Setting levels: control = 0, case = 1
\end{verbatim}

\begin{verbatim}
## Setting direction: controls < cases
\end{verbatim}

\begin{verbatim}
## Setting levels: control = 0, case = 1
\end{verbatim}

\begin{verbatim}
## Setting direction: controls < cases
\end{verbatim}

\begin{verbatim}
## Setting levels: control = 0, case = 1
\end{verbatim}

\begin{verbatim}
## Setting direction: controls < cases
\end{verbatim}

\begin{verbatim}
## Setting levels: control = 0, case = 1
\end{verbatim}

\begin{verbatim}
## Setting direction: controls < cases
\end{verbatim}

\begin{verbatim}
## Setting levels: control = 0, case = 1
\end{verbatim}

\begin{verbatim}
## Setting direction: controls < cases
\end{verbatim}

\begin{verbatim}
## Setting levels: control = 0, case = 1
\end{verbatim}

\begin{verbatim}
## Setting direction: controls < cases
\end{verbatim}

\begin{verbatim}
## Setting levels: control = 0, case = 1
\end{verbatim}

\begin{verbatim}
## Setting direction: controls < cases
\end{verbatim}

\begin{verbatim}
## Setting levels: control = 0, case = 1
\end{verbatim}

\begin{verbatim}
## Setting direction: controls < cases
\end{verbatim}

\begin{verbatim}
## Setting levels: control = 0, case = 1
\end{verbatim}

\begin{verbatim}
## Setting direction: controls < cases
\end{verbatim}

\begin{verbatim}
## Setting levels: control = 0, case = 1
\end{verbatim}

\begin{verbatim}
## Setting direction: controls < cases
\end{verbatim}

\begin{verbatim}
## Setting levels: control = 0, case = 1
\end{verbatim}

\begin{verbatim}
## Setting direction: controls < cases
\end{verbatim}

\begin{verbatim}
## Setting levels: control = 0, case = 1
\end{verbatim}

\begin{verbatim}
## Setting direction: controls < cases
\end{verbatim}

\begin{verbatim}
## Setting levels: control = 0, case = 1
\end{verbatim}

\begin{verbatim}
## Setting direction: controls < cases
\end{verbatim}

\begin{verbatim}
## Setting levels: control = 0, case = 1
\end{verbatim}

\begin{verbatim}
## Setting direction: controls < cases
\end{verbatim}

\begin{verbatim}
## Setting levels: control = 0, case = 1
\end{verbatim}

\begin{verbatim}
## Setting direction: controls < cases
\end{verbatim}

\begin{verbatim}
## Setting levels: control = 0, case = 1
\end{verbatim}

\begin{verbatim}
## Setting direction: controls < cases
\end{verbatim}

\begin{verbatim}
## Setting levels: control = 0, case = 1
\end{verbatim}

\begin{verbatim}
## Setting direction: controls < cases
\end{verbatim}

\begin{verbatim}
## Setting levels: control = 0, case = 1
\end{verbatim}

\begin{verbatim}
## Setting direction: controls < cases
\end{verbatim}

\begin{verbatim}
## Setting levels: control = 0, case = 1
\end{verbatim}

\begin{verbatim}
## Setting direction: controls < cases
\end{verbatim}

\begin{verbatim}
## Setting levels: control = 0, case = 1
\end{verbatim}

\begin{verbatim}
## Setting direction: controls < cases
\end{verbatim}

\begin{verbatim}
## Setting levels: control = 0, case = 1
\end{verbatim}

\begin{verbatim}
## Setting direction: controls < cases
\end{verbatim}

\begin{verbatim}
## Setting levels: control = 0, case = 1
\end{verbatim}

\begin{verbatim}
## Setting direction: controls < cases
\end{verbatim}

\begin{verbatim}
## Setting levels: control = 0, case = 1
\end{verbatim}

\begin{verbatim}
## Setting direction: controls < cases
\end{verbatim}

\begin{verbatim}
## Setting levels: control = 0, case = 1
\end{verbatim}

\begin{verbatim}
## Setting direction: controls < cases
\end{verbatim}

\begin{verbatim}
## Setting levels: control = 0, case = 1
\end{verbatim}

\begin{verbatim}
## Setting direction: controls < cases
\end{verbatim}

\begin{verbatim}
## Setting levels: control = 0, case = 1
\end{verbatim}

\begin{verbatim}
## Setting direction: controls < cases
\end{verbatim}

\begin{verbatim}
## Setting levels: control = 0, case = 1
\end{verbatim}

\begin{verbatim}
## Setting direction: controls < cases
\end{verbatim}

\begin{verbatim}
## Setting levels: control = 0, case = 1
\end{verbatim}

\begin{verbatim}
## Setting direction: controls < cases
\end{verbatim}

\begin{verbatim}
## Setting levels: control = 0, case = 1
\end{verbatim}

\begin{verbatim}
## Setting direction: controls < cases
\end{verbatim}

\begin{verbatim}
## Setting levels: control = 0, case = 1
\end{verbatim}

\begin{verbatim}
## Setting direction: controls < cases
\end{verbatim}

\begin{verbatim}
## Setting levels: control = 0, case = 1
\end{verbatim}

\begin{verbatim}
## Setting direction: controls < cases
\end{verbatim}

\begin{verbatim}
## Setting levels: control = 0, case = 1
\end{verbatim}

\begin{verbatim}
## Setting direction: controls < cases
\end{verbatim}

\begin{verbatim}
## Setting levels: control = 0, case = 1
\end{verbatim}

\begin{verbatim}
## Setting direction: controls < cases
\end{verbatim}

\begin{verbatim}
## Setting levels: control = 0, case = 1
\end{verbatim}

\begin{verbatim}
## Setting direction: controls < cases
\end{verbatim}

\begin{verbatim}
## Setting levels: control = 0, case = 1
\end{verbatim}

\begin{verbatim}
## Setting direction: controls < cases
\end{verbatim}

\begin{verbatim}
## Setting levels: control = 0, case = 1
\end{verbatim}

\begin{verbatim}
## Setting direction: controls < cases
\end{verbatim}

\begin{verbatim}
## Setting levels: control = 0, case = 1
\end{verbatim}

\begin{verbatim}
## Setting direction: controls < cases
\end{verbatim}

\begin{verbatim}
## Setting levels: control = 0, case = 1
\end{verbatim}

\begin{verbatim}
## Setting direction: controls < cases
\end{verbatim}

In our first generation of Z variables, we believed that Z and X
variables were independent and had no relationship. We then simulate the
correlation between the Z variable and the X variable. Here, we
constructed a parameter using the information of the X variable, and
used this parameter to generate many Z\_all variables. Next, we can
extract Z variables from Z\_all that meet the validity criteria.

\begin{Shaded}
\begin{Highlighting}[]
\NormalTok{Q }\OtherTok{\textless{}{-}} \FunctionTok{cbind}\NormalTok{(}\DecValTok{1}\NormalTok{,X1,X2,X3)}
\NormalTok{    theta }\OtherTok{\textless{}{-}} \FunctionTok{c}\NormalTok{(}\FloatTok{0.3}\NormalTok{,}\FloatTok{0.2}\NormalTok{,}\FloatTok{0.3}\NormalTok{,}\FloatTok{0.1}\NormalTok{)}
\NormalTok{    p\_z }\OtherTok{\textless{}{-}}\NormalTok{ Q}\SpecialCharTok{\%*\%}\NormalTok{theta}
    
\NormalTok{    p\_z[}\FunctionTok{which}\NormalTok{(p\_z }\SpecialCharTok{\textless{}} \DecValTok{0}\NormalTok{)] }\OtherTok{\textless{}{-}} \FloatTok{0.01}
\NormalTok{    p\_z[}\FunctionTok{which}\NormalTok{(p\_z }\SpecialCharTok{\textgreater{}} \DecValTok{1}\NormalTok{)] }\OtherTok{\textless{}{-}} \FloatTok{0.99}
    
\NormalTok{    Z\_all }\OtherTok{\textless{}{-}} \FunctionTok{rbinom}\NormalTok{(}\DecValTok{1000000}\NormalTok{, }\AttributeTok{size =} \DecValTok{1}\NormalTok{, }\AttributeTok{prob =}\NormalTok{ p\_z)}
    
\NormalTok{    Z }\OtherTok{\textless{}{-}} \FunctionTok{sample}\NormalTok{(Z\_all[Z\_all }\SpecialCharTok{==} \DecValTok{1}\NormalTok{], }\AttributeTok{size =}\NormalTok{ NN }\SpecialCharTok{*}\NormalTok{ prevalence\_Z[i])}
    
\NormalTok{    Z }\OtherTok{\textless{}{-}} \FunctionTok{c}\NormalTok{(Z, }\FunctionTok{rep}\NormalTok{(}\DecValTok{0}\NormalTok{, NN }\SpecialCharTok{{-}} \FunctionTok{length}\NormalTok{(Z)))}
    
\NormalTok{    Z }\OtherTok{\textless{}{-}} \FunctionTok{sample}\NormalTok{(Z)}
\end{Highlighting}
\end{Shaded}


\end{document}
